\PassOptionsToPackage{unicode=true}{hyperref} % options for packages loaded elsewhere
\PassOptionsToPackage{hyphens}{url}
%
\documentclass[]{article}
\usepackage{lmodern}
\usepackage{amssymb,amsmath}
\usepackage{ifxetex,ifluatex}
\usepackage{fixltx2e} % provides \textsubscript
\ifnum 0\ifxetex 1\fi\ifluatex 1\fi=0 % if pdftex
  \usepackage[T1]{fontenc}
  \usepackage[utf8]{inputenc}
  \usepackage{textcomp} % provides euro and other symbols
\else % if luatex or xelatex
  \usepackage{unicode-math}
  \defaultfontfeatures{Ligatures=TeX,Scale=MatchLowercase}
\fi
% use upquote if available, for straight quotes in verbatim environments
\IfFileExists{upquote.sty}{\usepackage{upquote}}{}
% use microtype if available
\IfFileExists{microtype.sty}{%
\usepackage[]{microtype}
\UseMicrotypeSet[protrusion]{basicmath} % disable protrusion for tt fonts
}{}
\IfFileExists{parskip.sty}{%
\usepackage{parskip}
}{% else
\setlength{\parindent}{0pt}
\setlength{\parskip}{6pt plus 2pt minus 1pt}
}
\usepackage{hyperref}
\hypersetup{
            pdftitle={R-Documentacion de Texto},
            pdfauthor={Curso estadistica descriptiva},
            pdfborder={0 0 0},
            breaklinks=true}
\urlstyle{same}  % don't use monospace font for urls
\usepackage[margin=1in]{geometry}
\usepackage{graphicx,grffile}
\makeatletter
\def\maxwidth{\ifdim\Gin@nat@width>\linewidth\linewidth\else\Gin@nat@width\fi}
\def\maxheight{\ifdim\Gin@nat@height>\textheight\textheight\else\Gin@nat@height\fi}
\makeatother
% Scale images if necessary, so that they will not overflow the page
% margins by default, and it is still possible to overwrite the defaults
% using explicit options in \includegraphics[width, height, ...]{}
\setkeys{Gin}{width=\maxwidth,height=\maxheight,keepaspectratio}
\usepackage[normalem]{ulem}
% avoid problems with \sout in headers with hyperref:
\pdfstringdefDisableCommands{\renewcommand{\sout}{}}
\setlength{\emergencystretch}{3em}  % prevent overfull lines
\providecommand{\tightlist}{%
  \setlength{\itemsep}{0pt}\setlength{\parskip}{0pt}}
\setcounter{secnumdepth}{0}
% Redefines (sub)paragraphs to behave more like sections
\ifx\paragraph\undefined\else
\let\oldparagraph\paragraph
\renewcommand{\paragraph}[1]{\oldparagraph{#1}\mbox{}}
\fi
\ifx\subparagraph\undefined\else
\let\oldsubparagraph\subparagraph
\renewcommand{\subparagraph}[1]{\oldsubparagraph{#1}\mbox{}}
\fi

% set default figure placement to htbp
\makeatletter
\def\fps@figure{htbp}
\makeatother

\usepackage{etoolbox}
\makeatletter
\providecommand{\subtitle}[1]{% add subtitle to \maketitle
  \apptocmd{\@title}{\par {\large #1 \par}}{}{}
}
\makeatother
% https://github.com/rstudio/rmarkdown/issues/337
\let\rmarkdownfootnote\footnote%
\def\footnote{\protect\rmarkdownfootnote}

% https://github.com/rstudio/rmarkdown/pull/252
\usepackage{titling}
\setlength{\droptitle}{-2em}

\pretitle{\vspace{\droptitle}\centering\huge}
\posttitle{\par}

\preauthor{\centering\large\emph}
\postauthor{\par}

\predate{\centering\large\emph}
\postdate{\par}

\title{R-Documentacion de Texto}
\author{Curso estadistica descriptiva}
\date{10/12/2019}

\begin{document}
\maketitle

\hypertarget{titulo-1}{%
\section{TITULO 1}\label{titulo-1}}

\hypertarget{titulo-2}{%
\subsection{TITULO 2}\label{titulo-2}}

\hypertarget{titulo-3}{%
\subsubsection{TITULO 3}\label{titulo-3}}

\hypertarget{titulo-4}{%
\paragraph{TITULO 4}\label{titulo-4}}

\hypertarget{titulo-5}{%
\subparagraph{TITULO 5}\label{titulo-5}}

\hypertarget{titulo-6}{%
\subparagraph{TITULO 6}\label{titulo-6}}

Esto es un texto llano. Podemos escribir sin ningún problema todo el
texto que queramos para acompañar a tanto el código en \texttt{R} como
las fórmulas en \LaTeX.

Esto sería una nueva línea de texto. Fijaros que para que separe los
párrafos necesitamos siempre almenos un doble intro.

Para colocar \emph{cursiva} podemos usar un asterisco: \emph{esto sería
un texto en cursiva} o bien usar una sola barra baja \emph{esto tambien
es cursiva}.

Para colocar \textbf{negrita} podemos usar un doble asterisco:
\textbf{esto es negrita} o bien una doble barra baja \textbf{esto
también es negrita}.

Los superíndices van con el \emph{sombrerito}: este curso vale por mi
nota\textsuperscript{2}. Ruber Hernandes\textsuperscript{profeonline}
\ldots{}

Para tachar una palabra usamos una doble tilde \sout{Las matemáticas son
un rollo}.

Para conocer más sobre los cursos de Juan Gabriel, podemos visitar su
\href{https://udemy.com/u/juangabriel}{perfil de Udemy}

endash: -- Y entonces Juan Gabriel dijo: \ldots{}.

emdash: \ldots{} --- como Juan Gabriel quería explicar.

elipsis: \ldots{}

ecuaciones en línea \(S = \pi\cdot r^2\).

imagen \includegraphics{../../teoria/Imgs/logo.png}

\begin{center}\rule{0.5\linewidth}{\linethickness}\end{center}

Aquí empezamos otra cosa

\begin{quote}
Esto es una cita en bloque
\end{quote}

\hypertarget{listas-no-ordenadas}{%
\subsubsection{Listas no ordenadas}\label{listas-no-ordenadas}}

\end{document}

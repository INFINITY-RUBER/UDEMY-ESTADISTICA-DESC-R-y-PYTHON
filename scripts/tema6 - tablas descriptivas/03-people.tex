\PassOptionsToPackage{unicode=true}{hyperref} % options for packages loaded elsewhere
\PassOptionsToPackage{hyphens}{url}
%
\documentclass[]{article}
\usepackage{lmodern}
\usepackage{amssymb,amsmath}
\usepackage{ifxetex,ifluatex}
\usepackage{fixltx2e} % provides \textsubscript
\ifnum 0\ifxetex 1\fi\ifluatex 1\fi=0 % if pdftex
  \usepackage[T1]{fontenc}
  \usepackage[utf8]{inputenc}
  \usepackage{textcomp} % provides euro and other symbols
\else % if luatex or xelatex
  \usepackage{unicode-math}
  \defaultfontfeatures{Ligatures=TeX,Scale=MatchLowercase}
\fi
% use upquote if available, for straight quotes in verbatim environments
\IfFileExists{upquote.sty}{\usepackage{upquote}}{}
% use microtype if available
\IfFileExists{microtype.sty}{%
\usepackage[]{microtype}
\UseMicrotypeSet[protrusion]{basicmath} % disable protrusion for tt fonts
}{}
\IfFileExists{parskip.sty}{%
\usepackage{parskip}
}{% else
\setlength{\parindent}{0pt}
\setlength{\parskip}{6pt plus 2pt minus 1pt}
}
\usepackage{hyperref}
\hypersetup{
            pdftitle={03-people},
            pdfauthor={Curso de Estadística Descriptiva},
            pdfborder={0 0 0},
            breaklinks=true}
\urlstyle{same}  % don't use monospace font for urls
\usepackage[margin=1in]{geometry}
\usepackage{color}
\usepackage{fancyvrb}
\newcommand{\VerbBar}{|}
\newcommand{\VERB}{\Verb[commandchars=\\\{\}]}
\DefineVerbatimEnvironment{Highlighting}{Verbatim}{commandchars=\\\{\}}
% Add ',fontsize=\small' for more characters per line
\usepackage{framed}
\definecolor{shadecolor}{RGB}{248,248,248}
\newenvironment{Shaded}{\begin{snugshade}}{\end{snugshade}}
\newcommand{\AlertTok}[1]{\textcolor[rgb]{0.94,0.16,0.16}{#1}}
\newcommand{\AnnotationTok}[1]{\textcolor[rgb]{0.56,0.35,0.01}{\textbf{\textit{#1}}}}
\newcommand{\AttributeTok}[1]{\textcolor[rgb]{0.77,0.63,0.00}{#1}}
\newcommand{\BaseNTok}[1]{\textcolor[rgb]{0.00,0.00,0.81}{#1}}
\newcommand{\BuiltInTok}[1]{#1}
\newcommand{\CharTok}[1]{\textcolor[rgb]{0.31,0.60,0.02}{#1}}
\newcommand{\CommentTok}[1]{\textcolor[rgb]{0.56,0.35,0.01}{\textit{#1}}}
\newcommand{\CommentVarTok}[1]{\textcolor[rgb]{0.56,0.35,0.01}{\textbf{\textit{#1}}}}
\newcommand{\ConstantTok}[1]{\textcolor[rgb]{0.00,0.00,0.00}{#1}}
\newcommand{\ControlFlowTok}[1]{\textcolor[rgb]{0.13,0.29,0.53}{\textbf{#1}}}
\newcommand{\DataTypeTok}[1]{\textcolor[rgb]{0.13,0.29,0.53}{#1}}
\newcommand{\DecValTok}[1]{\textcolor[rgb]{0.00,0.00,0.81}{#1}}
\newcommand{\DocumentationTok}[1]{\textcolor[rgb]{0.56,0.35,0.01}{\textbf{\textit{#1}}}}
\newcommand{\ErrorTok}[1]{\textcolor[rgb]{0.64,0.00,0.00}{\textbf{#1}}}
\newcommand{\ExtensionTok}[1]{#1}
\newcommand{\FloatTok}[1]{\textcolor[rgb]{0.00,0.00,0.81}{#1}}
\newcommand{\FunctionTok}[1]{\textcolor[rgb]{0.00,0.00,0.00}{#1}}
\newcommand{\ImportTok}[1]{#1}
\newcommand{\InformationTok}[1]{\textcolor[rgb]{0.56,0.35,0.01}{\textbf{\textit{#1}}}}
\newcommand{\KeywordTok}[1]{\textcolor[rgb]{0.13,0.29,0.53}{\textbf{#1}}}
\newcommand{\NormalTok}[1]{#1}
\newcommand{\OperatorTok}[1]{\textcolor[rgb]{0.81,0.36,0.00}{\textbf{#1}}}
\newcommand{\OtherTok}[1]{\textcolor[rgb]{0.56,0.35,0.01}{#1}}
\newcommand{\PreprocessorTok}[1]{\textcolor[rgb]{0.56,0.35,0.01}{\textit{#1}}}
\newcommand{\RegionMarkerTok}[1]{#1}
\newcommand{\SpecialCharTok}[1]{\textcolor[rgb]{0.00,0.00,0.00}{#1}}
\newcommand{\SpecialStringTok}[1]{\textcolor[rgb]{0.31,0.60,0.02}{#1}}
\newcommand{\StringTok}[1]{\textcolor[rgb]{0.31,0.60,0.02}{#1}}
\newcommand{\VariableTok}[1]{\textcolor[rgb]{0.00,0.00,0.00}{#1}}
\newcommand{\VerbatimStringTok}[1]{\textcolor[rgb]{0.31,0.60,0.02}{#1}}
\newcommand{\WarningTok}[1]{\textcolor[rgb]{0.56,0.35,0.01}{\textbf{\textit{#1}}}}
\usepackage{graphicx,grffile}
\makeatletter
\def\maxwidth{\ifdim\Gin@nat@width>\linewidth\linewidth\else\Gin@nat@width\fi}
\def\maxheight{\ifdim\Gin@nat@height>\textheight\textheight\else\Gin@nat@height\fi}
\makeatother
% Scale images if necessary, so that they will not overflow the page
% margins by default, and it is still possible to overwrite the defaults
% using explicit options in \includegraphics[width, height, ...]{}
\setkeys{Gin}{width=\maxwidth,height=\maxheight,keepaspectratio}
\setlength{\emergencystretch}{3em}  % prevent overfull lines
\providecommand{\tightlist}{%
  \setlength{\itemsep}{0pt}\setlength{\parskip}{0pt}}
\setcounter{secnumdepth}{0}
% Redefines (sub)paragraphs to behave more like sections
\ifx\paragraph\undefined\else
\let\oldparagraph\paragraph
\renewcommand{\paragraph}[1]{\oldparagraph{#1}\mbox{}}
\fi
\ifx\subparagraph\undefined\else
\let\oldsubparagraph\subparagraph
\renewcommand{\subparagraph}[1]{\oldsubparagraph{#1}\mbox{}}
\fi

% set default figure placement to htbp
\makeatletter
\def\fps@figure{htbp}
\makeatother

\usepackage{booktabs}
\usepackage{longtable}
\usepackage{array}
\usepackage{multirow}
\usepackage{wrapfig}
\usepackage{float}
\usepackage{colortbl}
\usepackage{pdflscape}
\usepackage{tabu}
\usepackage{threeparttable}
\usepackage{threeparttablex}
\usepackage[normalem]{ulem}
\usepackage{makecell}
\usepackage{xcolor}

\title{03-people}
\author{Curso de Estadística Descriptiva}
\date{24/12/2018}

\begin{document}
\maketitle

\hypertarget{ejemplo-de-color-de-ojos-y-de-pelo}{%
\section{Ejemplo de color de ojos y de
pelo}\label{ejemplo-de-color-de-ojos-y-de-pelo}}

\begin{Shaded}
\begin{Highlighting}[]
\NormalTok{HairEyeColor}
\end{Highlighting}
\end{Shaded}

\begin{verbatim}
## , , Sex = Male
## 
##        Eye
## Hair    Brown Blue Hazel Green
##   Black    32   11    10     3
##   Brown    53   50    25    15
##   Red      10   10     7     7
##   Blond     3   30     5     8
## 
## , , Sex = Female
## 
##        Eye
## Hair    Brown Blue Hazel Green
##   Black    36    9     5     2
##   Brown    66   34    29    14
##   Red      16    7     7     7
##   Blond     4   64     5     8
\end{verbatim}

\begin{Shaded}
\begin{Highlighting}[]
\KeywordTok{sum}\NormalTok{(HairEyeColor) ->}\StringTok{ }\NormalTok{total}
\end{Highlighting}
\end{Shaded}

El total de individuos de la tabla de datos es 592.

\begin{Shaded}
\begin{Highlighting}[]
\KeywordTok{prop.table}\NormalTok{(HairEyeColor, }\DataTypeTok{margin =} \DecValTok{3}\NormalTok{)}\CommentTok{# frecuencia marginal}
\end{Highlighting}
\end{Shaded}

\begin{verbatim}
## , , Sex = Male
## 
##        Eye
## Hair          Brown        Blue       Hazel       Green
##   Black 0.114695341 0.039426523 0.035842294 0.010752688
##   Brown 0.189964158 0.179211470 0.089605735 0.053763441
##   Red   0.035842294 0.035842294 0.025089606 0.025089606
##   Blond 0.010752688 0.107526882 0.017921147 0.028673835
## 
## , , Sex = Female
## 
##        Eye
## Hair          Brown        Blue       Hazel       Green
##   Black 0.115015974 0.028753994 0.015974441 0.006389776
##   Brown 0.210862620 0.108626198 0.092651757 0.044728435
##   Red   0.051118211 0.022364217 0.022364217 0.022364217
##   Blond 0.012779553 0.204472843 0.015974441 0.025559105
\end{verbatim}

\begin{Shaded}
\begin{Highlighting}[]
\KeywordTok{prop.table}\NormalTok{(HairEyeColor, }\DataTypeTok{margin =} \KeywordTok{c}\NormalTok{(}\DecValTok{1}\NormalTok{,}\DecValTok{2}\NormalTok{)) }\CommentTok{# la marginal repartido por ojos}
\end{Highlighting}
\end{Shaded}

\begin{verbatim}
## , , Sex = Male
## 
##        Eye
## Hair        Brown      Blue     Hazel     Green
##   Black 0.4705882 0.5500000 0.6666667 0.6000000
##   Brown 0.4453782 0.5952381 0.4629630 0.5172414
##   Red   0.3846154 0.5882353 0.5000000 0.5000000
##   Blond 0.4285714 0.3191489 0.5000000 0.5000000
## 
## , , Sex = Female
## 
##        Eye
## Hair        Brown      Blue     Hazel     Green
##   Black 0.5294118 0.4500000 0.3333333 0.4000000
##   Brown 0.5546218 0.4047619 0.5370370 0.4827586
##   Red   0.6153846 0.4117647 0.5000000 0.5000000
##   Blond 0.5714286 0.6808511 0.5000000 0.5000000
\end{verbatim}

\begin{Shaded}
\begin{Highlighting}[]
\KeywordTok{aperm}\NormalTok{(HairEyeColor, }\DataTypeTok{perm =} \KeywordTok{c}\NormalTok{(}\StringTok{"Sex"}\NormalTok{, }\StringTok{"Hair"}\NormalTok{, }\StringTok{"Eye"}\NormalTok{)) }\CommentTok{# cambiar el orden de las Columnas}
\end{Highlighting}
\end{Shaded}

\begin{verbatim}
## , , Eye = Brown
## 
##         Hair
## Sex      Black Brown Red Blond
##   Male      32    53  10     3
##   Female    36    66  16     4
## 
## , , Eye = Blue
## 
##         Hair
## Sex      Black Brown Red Blond
##   Male      11    50  10    30
##   Female     9    34   7    64
## 
## , , Eye = Hazel
## 
##         Hair
## Sex      Black Brown Red Blond
##   Male      10    25   7     5
##   Female     5    29   7     5
## 
## , , Eye = Green
## 
##         Hair
## Sex      Black Brown Red Blond
##   Male       3    15   7     8
##   Female     2    14   7     8
\end{verbatim}

\begin{Shaded}
\begin{Highlighting}[]
\KeywordTok{library}\NormalTok{(kableExtra)}\CommentTok{# para mirar la info completa forma tabla}
\KeywordTok{kable}\NormalTok{(HairEyeColor)}
\end{Highlighting}
\end{Shaded}

\begin{tabular}{l|l|l|r}
\hline
Hair & Eye & Sex & Freq\\
\hline
Black & Brown & Male & 32\\
\hline
Brown & Brown & Male & 53\\
\hline
Red & Brown & Male & 10\\
\hline
Blond & Brown & Male & 3\\
\hline
Black & Blue & Male & 11\\
\hline
Brown & Blue & Male & 50\\
\hline
Red & Blue & Male & 10\\
\hline
Blond & Blue & Male & 30\\
\hline
Black & Hazel & Male & 10\\
\hline
Brown & Hazel & Male & 25\\
\hline
Red & Hazel & Male & 7\\
\hline
Blond & Hazel & Male & 5\\
\hline
Black & Green & Male & 3\\
\hline
Brown & Green & Male & 15\\
\hline
Red & Green & Male & 7\\
\hline
Blond & Green & Male & 8\\
\hline
Black & Brown & Female & 36\\
\hline
Brown & Brown & Female & 66\\
\hline
Red & Brown & Female & 16\\
\hline
Blond & Brown & Female & 4\\
\hline
Black & Blue & Female & 9\\
\hline
Brown & Blue & Female & 34\\
\hline
Red & Blue & Female & 7\\
\hline
Blond & Blue & Female & 64\\
\hline
Black & Hazel & Female & 5\\
\hline
Brown & Hazel & Female & 29\\
\hline
Red & Hazel & Female & 7\\
\hline
Blond & Hazel & Female & 5\\
\hline
Black & Green & Female & 2\\
\hline
Brown & Green & Female & 14\\
\hline
Red & Green & Female & 7\\
\hline
Blond & Green & Female & 8\\
\hline
\end{tabular}

\begin{Shaded}
\begin{Highlighting}[]
\KeywordTok{library}\NormalTok{(xtable) }\CommentTok{# para tablas para  2 demenciones}
\NormalTok{sex =}\StringTok{ }\KeywordTok{factor}\NormalTok{(}\KeywordTok{c}\NormalTok{(}\StringTok{"H"}\NormalTok{, }\StringTok{"M"}\NormalTok{, }\StringTok{"M"}\NormalTok{, }\StringTok{"M"}\NormalTok{, }\StringTok{"H"}\NormalTok{, }\StringTok{"H"}\NormalTok{, }\StringTok{"M"}\NormalTok{, }\StringTok{"M"}\NormalTok{))}
\NormalTok{ans =}\StringTok{ }\KeywordTok{factor}\NormalTok{(}\KeywordTok{c}\NormalTok{(}\StringTok{"S"}\NormalTok{, }\StringTok{"N"}\NormalTok{, }\StringTok{"S"}\NormalTok{, }\StringTok{"S"}\NormalTok{, }\StringTok{"S"}\NormalTok{, }\StringTok{"N"}\NormalTok{, }\StringTok{"N"}\NormalTok{, }\StringTok{"S"}\NormalTok{))}

\KeywordTok{xtable}\NormalTok{(}\KeywordTok{table}\NormalTok{(sex, ans))}
\end{Highlighting}
\end{Shaded}

\% latex table generated in R 3.4.4 by xtable 1.8-4 package \% Sun Feb
16 13:42:01 2020

\begin{table}[ht]
\centering
\begin{tabular}{rrr}
  \hline
 & N & S \\ 
  \hline
H &   1 &   2 \\ 
  M &   2 &   3 \\ 
   \hline
\end{tabular}
\end{table}

\end{document}
